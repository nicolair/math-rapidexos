\begin{align*}
 A \text{ de coordonn�es } (0,-3,-2) & &
\overrightarrow u \text{ de coordonn�es } (1,2,3)
\end{align*}
